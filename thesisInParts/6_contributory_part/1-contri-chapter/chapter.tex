
% \[
% \begin{bmatrix}
% \texttt{19} & \texttt{a0} & \texttt{9a} & \texttt{e9} \\
% \texttt{3d} & \textcolor{red}{\boxed{\texttt{f4}}} & \texttt{c6} & \texttt{f8} \\
% \texttt{e3} & \texttt{e2} & \texttt{8d} & \texttt{48} \\
% \texttt{be} & \texttt{2b} & \texttt{2a} & \texttt{08}
% \end{bmatrix}
% \]

The Advanced Encryption Standard (AES) has stood resilient for decades, becoming a cornerstone in modern cryptographic systems. However, side-channel attacks and fault analysis have emerged as powerful methods to compromise even the most robust ciphers, not by breaking the algorithm itself, but by exploiting its implementation. One such sophisticated technique is the Diagonal Fault Attack (DFA), a targeted fault analysis method designed to extract cryptographic keys from AES by injecting faults into specific portions of the cipher’s internal state.

In this chapter, we reproduce the diagonal fault attack on AES as presented by Dhiman et al. in their 2009 paper titled "A Diagonal Fault Attack on the Advanced Encryption Standard" \cite{Saha2009ADF}. Their work introduced a fault model where a single fault injected into a diagonal of the AES state matrix during the final rounds of encryption enables efficient recovery of the secret key.

To validate their proposed attack in a practical setting, we implemented the diagonal fault injection using two physical fault methods: clock glitching and voltage glitching, both facilitated by the ChipWhisperer Lite (CWLite) platform. Our goal is to observe diagonal fault patterns in the faulty ciphertexts and subsequently perform key recovery using the methodology described in the original paper.

This experimental reproduction helps bridge the gap between theoretical fault models and their real-world applicability, emphasizing the feasibility and efficiency of diagonal fault attacks under controlled glitching conditions.

\section{Four Diagonals in the AES State Matrix}

In the AES algorithm, the internal 128-bit state is arranged as a 4×4 matrix of bytes:


\[
\begin{array}{cc}
\textbf{Diagonal 0}~[D_0] & \textbf{Diagonal 1}~[D_1] \\[1ex]
\begin{array}{|c|c|c|c|}
\hline
\fcolorbox{red}{red!20}{$a_{00}$} & a_{01} & a_{02} & a_{03} \\
\hline
a_{10} & \fcolorbox{red}{red!20}{$a_{11}$} & a_{12} & a_{13} \\
\hline
a_{20} & a_{21} & \fcolorbox{red}{red!20}{$a_{22}$} & a_{23} \\
\hline
a_{30} & a_{31} & a_{32} & \fcolorbox{red}{red!20}{$a_{33}$} \\
\hline
\end{array}
&
\begin{array}{|c|c|c|c|}
\hline
a_{00} & \fcolorbox{blue}{blue!20}{$a_{01}$} & a_{02} & a_{03} \\
\hline
a_{10} & a_{11} & \fcolorbox{blue}{blue!20}{$a_{12}$} & a_{13} \\
\hline
a_{20} & a_{21} & a_{22} & \fcolorbox{blue}{blue!20}{$a_{23}$} \\
\hline
\fcolorbox{blue}{blue!20}{$a_{30}$} & a_{31} & a_{32} & a_{33} \\
\hline
\end{array}
\\[6ex]
\textbf{Diagonal 2}~[D_2] & \textbf{Diagonal 3}~[D_3] \\[1ex]
\begin{array}{|c|c|c|c|}
\hline
a_{00} & a_{01} & \fcolorbox{green!50!black}{green!20}{$a_{02}$} & a_{03} \\
\hline
a_{10} & a_{11} & a_{12} & \fcolorbox{green!50!black}{green!20}{$a_{13}$} \\
\hline
\fcolorbox{green!50!black}{green!20}{$a_{20}$} & a_{21} & a_{22} & a_{23} \\
\hline
a_{30} & \fcolorbox{green!50!black}{green!20}{$a_{31}$} & a_{32} & a_{33} \\
\hline
\end{array}
&
\begin{array}{|c|c|c|c|}
\hline
a_{00} & a_{01} & a_{02} & \fcolorbox{orange!80!black}{orange!20}{$a_{03}$} \\
\hline
\fcolorbox{orange!80!black}{orange!20}{$a_{10}$} & a_{11} & a_{12} & a_{13} \\
\hline
a_{20} & \fcolorbox{orange!80!black}{orange!20}{$a_{21}$} & a_{22} & a_{23} \\
\hline
a_{30} & a_{31} & \fcolorbox{orange!80!black}{orange!20}{$a_{32}$} & a_{33} \\
\hline
\end{array}
\end{array}
\]


In the above matrices, each element \(a_{ij}\) represents the entry in the \(i^{\text{th}}\) row and \(j^{\text{th}}\) column of a \(4 \times 4\) state matrix. This state matrix is commonly used in AES (Advanced Encryption Standard) to represent the internal data at various stages of the encryption or decryption process. The highlighted elements correspond to the entries located along one of the four diagonals of the matrix, indexed as \(D_0, D_1, D_2,\) and \(D_3\), respectively.

\section{Diagonal Fault Attack on AES-128 with Fault Injection at Round 8}
In AES-128, the encryption process consists of 10 rounds, with each round transforming a 4×4 byte matrix called the state. Each round includes operations such as SubBytes, ShiftRows, MixColumns, and AddRoundKey. Notably, the AES state is updated in a predictable pattern, and faults injected in earlier rounds propagate through subsequent transformations in a structured way.
Step-by-Step Fault Propagation:

\textbf{Fault Introduction (Start of Round 8):}
A fault is injected into one or more bytes of one or more diagonals in the AES state matrix. This occurs just after the \texttt{AddRoundKey} operation of Round 7 and before the \texttt{SubBytes} transformation of Round 8.

\textbf{SubBytes (Non-linear step):}
Each byte in the state, including any faulty bytes, undergoes substitution through the AES S-box. This non-linear transformation alters the faulty bytes unpredictably. However, at this stage, the fault remains localized to the originally affected bytes.

\textbf{ShiftRows (Byte reordering):}
In this step, each row of the state matrix is cyclically shifted by a specific offset. As a result, the previously localized fault bytes—initially within a single diagonal—are now redistributed into different columns. This marks the beginning of spatial fault propagation.

\textbf{MixColumns (Diffusion step):}
This transformation applies a fixed matrix multiplication over a finite field to each column of the state. Consequently, a single faulty byte in any column results in the corruption of all four bytes within that column. At this stage, the fault diffuses significantly across the state.

\textbf{AddRoundKey :}
The transformed (and now faulty) state is XORed with the Round 8 key. The fault remains embedded and is now more spread out across the matrix.

\textbf{Rounds 9 and 10:}
The corrupted state continues to evolve through the remaining AES rounds. Round 9 further propagates the fault via \texttt{SubBytes}, \texttt{ShiftRows}, and \texttt{MixColumns}. However, Round 10 omits the \texttt{MixColumns} step, causing the fault pattern to stabilize. The final faulty ciphertext thus reflects this structured propagation, which can be analyzed to extract internal state information.
\ref{fig:powertrace} is the Power Trace of the AES-128 running in the integrated target of CWLite-ARM. This clearly shows the AES128 rounds:
\begin{figure}[h]
    \centering
    \includegraphics[width=0.9\linewidth, height=0.5\linewidth]{/home/cp07/mymtechthesis/thesisInParts/images/aes128_powertrace.png}
    \caption{AES128 Power Trace of CWLite-ARM}
    \label{fig:powertrace}
\end{figure}
\section{Fault Injection at the Start of Round 8 using CW-Lite}

To analyze the behavior of AES under fault conditions, a fault can be intentionally introduced at the beginning of Round 8 using the ChipWhisperer-Lite (CW-Lite) platform. Two common techniques supported by CW-Lite for fault injection are voltage glitching and clock glitching. These methods allow precise control over when and how a fault is introduced into the target device running AES.

\subsection{Using Clock Glitching}
We have identified a probable location(5080 to 5120 samples) of the starting of 8th round from the power trace of the AES-128 running in the integrated target of CWLite-ARM. Then we have set the parametsers for voltage glitching as follows:
\begin{verbatim}
    scope.glitch.clk_src = "clkgen"
    scope.glitch.output = "clock_xor"
    scope.glitch.trigger_src = "ext_single"
    scope.glitch.repeat = 1
    scope.io.hs2 = "glitch"
\end{verbatim}
\subsubsection{Result}
A glitch was introduced at location 5086 samples, which corresponds to the start of Round 8. The parametsers 
\begin{verbatim}
    scope.glitch.offset= 10
    scope.glitch.width= 3
\end{verbatim}
were responsible for this fault injection. This resulted obtaining a faulty ciphertext b'56 9b 6f 66 c9 41 96 f7 9c f1 49 44 29 13 04 e4' where the correct ciphertext without any fault would be b'f5 d3 d5 85 03 b9 69 9d e7 85 89 5a 96 fd ba af'.

\subsubsection{Fault Propagation \& Analysis}
We have one correct ciphertext and one faulty ciphertext after the fault injection. As the key was already known to us we have compared the two ciphertexts to analyze the fault propagation using the AES-128 Decryption method. The following tables show the state of the AES matrix at various stages of the encryption process, both with and without the fault.


\subsubsection{After The Fault Injection At Starting of $8^{\text{th}}$ Round :}
\[
\begin{array}{ccc}
\textbf{Without Fault} & \textbf{With Fault } & \textbf{Difference } \\[1ex]
\begin{array}{|c|c|c|c|}
\hline
\texttt{96} & \texttt{e9} & \texttt{e9} & \texttt{3c} \\
\hline
\texttt{71} & \texttt{87} & \texttt{61} & \texttt{89} \\
\hline
\texttt{6a} & \texttt{91} & \texttt{04} & \texttt{13} \\
\hline
\texttt{e4} & \texttt{c7} & \texttt{90} & \texttt{ff} \\
\hline
\end{array} 
&

\begin{array}{|c|c|c|c|}
    \hline
    \texttt{96} & \texttt{e9} & \texttt{e9} & \texttt{3c} \\
    \hline
    \texttt{71} & \texttt{a3} & \texttt{61} & \texttt{89} \\
    \hline
    \texttt{6a} & \texttt{91} & \texttt{04} & \texttt{13} \\
    \hline
    \texttt{e4} & \texttt{c7} & \texttt{90} & \texttt{ff} \\
    \hline
    \end{array}

    &

\begin{array}{|c|c|c|c|}
    \hline
    \texttt{00} & \texttt{00} & \texttt{00} &\texttt{00} \\
    \hline
    {\texttt{00}} & \fcolorbox{red}{red!20}{\texttt{24}} & \texttt{00} & \texttt{00} \\
    \hline
    \texttt{00} & \texttt{00} & \texttt{00} & \texttt{00} \\
    \hline
    \texttt{00} & \texttt{00} & \texttt{00} & \texttt{00} \\
    \hline
    \end{array}
\end{array}
\]

\subsubsection{After The Completion of $8^{\text{th}}$ Round(SubByte,ShiftRow,MixColumn,AddRoundkey(8)) :}
\[
\begin{array}{ccc}
\textbf{Without Fault} & \textbf{With Fault } & \textbf{Difference } \\[1ex]
\begin{array}{|c|c|c|c|}
\hline
\texttt{0c} & \texttt{b7} & \texttt{3b} & \texttt{ad} \\
\hline
\texttt{77} & \texttt{b8} & \texttt{a0} & \texttt{c3} \\
\hline
\texttt{31} & \texttt{0a} & \texttt{19} & \texttt{d8} \\
\hline
\texttt{43} & \texttt{b0} & \texttt{70} & \texttt{eb} \\
\hline
\end{array} 
&

\begin{array}{|c|c|c|c|}
    \hline
    \texttt{2b} & \texttt{b7} & \texttt{3b} & \texttt{ad} \\
    \hline
    \texttt{4d} & \texttt{b8} & \texttt{a0} & \texttt{c3} \\
    \hline
    \texttt{2c} & \texttt{0a} & \texttt{19} & \texttt{d8} \\
    \hline
    \texttt{5e} & \texttt{b0} & \texttt{70} & \texttt{eb} \\
    \hline
    \end{array}

    &

\begin{array}{|c|c|c|c|}
    \hline
    \fcolorbox{red}{red!20}{\texttt{27}} & \texttt{00} & \texttt{00} & \texttt{00} \\
    \hline
    \fcolorbox{red}{red!20}{\texttt{3a}} & \texttt{00} & \texttt{00} & \texttt{00}  \\
    \hline
    \fcolorbox{red}{red!20}{\texttt{1d}} & \texttt{00} & \texttt{00} & \texttt{00}  \\
    \hline
    \fcolorbox{red}{red!20}{\texttt{1d}} & \texttt{00} & \texttt{00} & \texttt{00}  \\
    \hline
    \end{array}
\end{array}
\]

\subsubsection{After The Completion of $9^{\text{th}}$ Round(SubByte,ShiftRow,MixColumn,AddRoundkey(9)) :}
\[
\begin{array}{ccc}
\textbf{Without Fault} & \textbf{With Fault } & \textbf{Difference } \\[1ex]
\begin{array}{|c|c|c|c|}
    \hline
    \texttt{c2} & \texttt{10} & \texttt{a5} & \texttt{54} \\
    \hline
    \texttt{df} & \texttt{31} & \texttt{da} & \texttt{c0} \\
    \hline
    \texttt{67} & \texttt{79} & \texttt{42} & \texttt{5d} \\
    \hline
    \texttt{9b} & \texttt{74} & \texttt{40} & \texttt{fa} \\
    \hline
\end{array} 
&

\begin{array}{|c|c|c|c|}
    \hline
    \texttt{dc} & \texttt{52} & \texttt{13} & \texttt{6e} \\
    \hline
    \texttt{d0} & \texttt{73} & \texttt{1b} & \texttt{ec} \\
    \hline
    \texttt{68} & \texttt{bf} & \texttt{35} & \texttt{4b} \\
    \hline
    \texttt{8a} & \texttt{f0} & \texttt{f6} & \texttt{ec} \\
    \hline
    \end{array}

    &

    \begin{array}{|c|c|c|c|}
        \hline
        \fcolorbox{red}{red!20}{\texttt{1e}} & \fcolorbox{red}{red!20}{\texttt{42}} & \fcolorbox{red}{red!20}{\texttt{b6}} & \fcolorbox{red}{red!20}{\texttt{3a}} \\
        \hline
        \fcolorbox{red}{red!20}{\texttt{0f}} & \fcolorbox{red}{red!20}{\texttt{42}} & \fcolorbox{red}{red!20}{\texttt{c1}} & \fcolorbox{red}{red!20}{\texttt{2c}} \\
        \hline
        \fcolorbox{red}{red!20}{\texttt{0f}} & \fcolorbox{red}{red!20}{\texttt{c6}} & \fcolorbox{red}{red!20}{\texttt{77}} & \fcolorbox{red}{red!20}{\texttt{16}} \\
        \hline
        \fcolorbox{red}{red!20}{\texttt{11}} & \fcolorbox{red}{red!20}{\texttt{84}} & \fcolorbox{red}{red!20}{\texttt{b6}} & \fcolorbox{red}{red!20}{\texttt{16}} \\
        \hline
    \end{array}
\end{array}
\]
\subsubsection{After The Completion of $10^{\text{th}}$ Round(SubByte,ShiftRow,AddRoundkey(10)) :}
\[
\begin{array}{ccc}
\textbf{Without Fault ($ct$)} & \textbf{With Fault ($ct_f$)} & \textbf{Difference } \\[1ex]
\begin{array}{|c|c|c|c|}
    \hline
    \texttt{f5} & \texttt{03} & \texttt{e7} & \texttt{96} \\
    \hline
    \texttt{d3} & \texttt{b9} & \texttt{85} & \texttt{fd} \\
    \hline
    \texttt{d5} & \texttt{69} & \texttt{89} & \texttt{ba} \\
    \hline
    \texttt{85} & \texttt{9d} & \texttt{5a} & \texttt{af} \\
    \hline
\end{array} 
&

\begin{array}{|c|c|c|c|}
    \hline
    \texttt{56} & \texttt{c9} & \texttt{9c} & \texttt{29} \\
    \hline
    \texttt{9b} & \texttt{41} & \texttt{f1} & \texttt{13} \\
    \hline
    \texttt{6f} & \texttt{96} & \texttt{49} & \texttt{04} \\
    \hline
    \texttt{66} & \texttt{f7} & \texttt{44} & \texttt{e4} \\
    \hline
\end{array}

    &

    \begin{array}{|c|c|c|c|}
        \hline
        \fcolorbox{red}{red!20}{\texttt{a3}} & \fcolorbox{red}{red!20}{\texttt{ca}} & \fcolorbox{red}{red!20}{\texttt{7b}} & \fcolorbox{red}{red!20}{\texttt{bf}} \\
        \hline
        \fcolorbox{red}{red!20}{\texttt{48}} & \fcolorbox{red}{red!20}{\texttt{f8}} & \fcolorbox{red}{red!20}{\texttt{74}} & \fcolorbox{red}{red!20}{\texttt{ee}} \\
        \hline
        \fcolorbox{red}{red!20}{\texttt{ba}} & \fcolorbox{red}{red!20}{\texttt{ff}} & \fcolorbox{red}{red!20}{\texttt{c0}} & \fcolorbox{red}{red!20}{\texttt{be}} \\
        \hline
        \fcolorbox{red}{red!20}{\texttt{e3}} & \fcolorbox{red}{red!20}{\texttt{6a}} & \fcolorbox{red}{red!20}{\texttt{1e}} & \fcolorbox{red}{red!20}{\texttt{4b}} \\
        \hline
    \end{array}
    
\end{array}
\]
Here we notice that the fault was injected at 1 bytes of the $D_0$ diagonal of the state matrix of 8th round. According to the fault model proposed by Dhiman et al.,\cite{Saha2009ADF} the fault in the $D_0$ diagonal of the state matrix at the start of Round 8 propagates through the subsequent rounds, affecting the final ciphertext. The differences in the ciphertexts before and after the fault injection reveal how the fault has altered specific bytes, which can be exploited to recover parts of the secret key. 
If we consider Byte inter-relations at the end of ninth round corresponding to $D_0$ diagonal, we can express the following relation:
\[
\begin{array}{ccc}
    \textbf{After 8th Round} & \textbf{After 9th Round} & \textbf{After 10th Round Shift Row} \\[1ex]
    \begin{array}{|c|c|c|c|}
        \hline
        \texttt{$f_1$} & \texttt{ } & \texttt{ } &  \texttt{ }\\
        \hline
        \texttt{$f_2$} & \texttt{ } & \texttt{ } &  \texttt{ }  \\
        \hline
        \texttt{$f_3$} & \texttt{ } & \texttt{ } &  \texttt{ }  \\
        \hline
        \texttt{$f_4$}& \texttt{ } & \texttt{ } &  \texttt{ }  \\
        \hline
    \end{array}
&

\begin{array}{|c|c|c|c|}
    \hline
    \texttt{2$f_1$} & \texttt{$f_4$} & \texttt{$f_3$} & \texttt{3$f_2$} \\
    \hline
    \texttt{$f_1$} & \texttt{$f_4$} & \texttt{3$f_3$} & \texttt{2$f_2$} \\
    \hline
    \texttt{$f_1$} & \texttt{3$f_4$} & \texttt{2$f_3$} & \texttt{$f_2$} \\
    \hline
    \texttt{3$f_1$} & \texttt{2$f_4$} & \texttt{$f_3$} & \texttt{$f_2$} \\
    \hline
\end{array}

    &

    \begin{array}{|c|c|c|c|}
        \hline
        \texttt{2$f_1$} & \texttt{$f_4$} & \texttt{$f_3$} & \texttt{3$f_2$} \\
        \hline
        \texttt{$f_4$} & \texttt{3$f_3$} & \texttt{2$f_2$} & \texttt{$f_1$}\\
        \hline
        \texttt{2$f_3$} & \texttt{$f_2$} & \texttt{$f_1$} & \texttt{3$f_4$} \\
        \hline
        \texttt{$f_2$} & \texttt{3$f_1$} & \texttt{2$f_4$} & \texttt{$f_3$} \\
        \hline
    \end{array}
    
\end{array}
\]
If we represent the $10^{th}$ round key as \(K_{10}\), it can be expressed as:
\[
    \begin{array}{|c|c|c|c|}
        \hline
        k_{00} & k_{01} & k_{02} & k_{03} \\
        \hline
        k_{10} & k_{11} & k_{12} & k_{13} \\
        \hline
        k_{20} & k_{21} & k_{22} & k_{23} \\
        \hline
        k_{30} & k_{31} & k_{32} & k_{33} \\
        \hline
        \end{array}
\]
Now we can frame 3 equations based on the differences in the ciphertexts before and after the fault injection to guess the key bytes $K_{00}$,$K_{13}$,$K_{22}$ and $K_{31}$ . The equations are as follows:
\begin{align*}
    \text{ISB}(\text{ct}[0] \oplus K_{00}) \oplus \text{ISB}(\text{ct}_f[0] \oplus K_{00}) 
    &= \text{mul2} \left( \text{ISB}(\text{ct}[13] \oplus K_{13}) \oplus \text{ISB}(\text{ct}_f[13] \oplus K_{13}) \right)\\
    \text{ISB}(\text{ct}[13] \oplus K_{13}) \oplus \text{ISB}(\text{ct}_f[13] \oplus K_{13}) 
    &= \text{ISB}(\text{ct}[10] \oplus K_{22}) \oplus \text{ISB}(\text{ct}_f[10] \oplus K_{22})\\
    \text{ISB}(\text{ct}[7] \oplus K_{31}) \oplus \text{ISB}(\text{ct}_f[7] \oplus K_{31}) 
    &= \text{mul3} \left( \text{ISB}(\text{ct}[13] \oplus K_{13}) \oplus \text{ISB}(\text{ct}_f[13] \oplus K_{13}) \right)
    \end{align*}

Similarly for keybytes $K_{01}$,$K_{12}$,$K_{23}$ and $K_{30}$ we can frame the following equations:
\begin{align*}
    \text{ISB}(\text{ct}[11] \oplus K_{32}) \oplus \text{ISB}(\text{ct}_f[11] \oplus K_{32}) &= \text{mul2} \cdot \left( \text{ISB}(\text{ct}[4] \oplus K_{01}) \oplus \text{ISB}(\text{ct}_f[4] \oplus K_{01}) \right) \\
    \text{ISB}(\text{ct}[1] \oplus K_{10}) \oplus \text{ISB}(\text{ct}_f[1] \oplus K_{10}) &= \text{ISB}(\text{ct}[4] \oplus K_{01}) \oplus \text{ISB}(\text{ct}_f[4] \oplus K_{01}) \\
    \text{ISB}(\text{ct}[14] \oplus K_{23}) \oplus \text{ISB}(\text{ct}_f[14] \oplus K_{23}) &= \text{mul3} \cdot \left( \text{ISB}(\text{ct}[4] \oplus K_{1}) \oplus \text{ISB}(\text{ct}_f[4] \oplus K_{1}) \right)
    \end{align*}

For keybytes $K_{02}$,$K_{11}$,$K_{20}$ and $K_{33}$ we can frame the following equations:
\begin{align*}
    \text{ISB}(\text{ct}[2] \oplus K_{20}) \oplus \text{ISB}(\text{ct}_f[2] \oplus K_{20}) &= \text{mul2} \cdot \left( \text{ISB}(\text{ct}[8] \oplus K_{02}) \oplus \text{ISB}(\text{ct}_f[8] \oplus K_{02}) \right) \\
    \text{ISB}(\text{ct}[15] \oplus K_{33}) \oplus \text{ISB}(\text{ct}_f[15] \oplus K_{33}) &= \text{ISB}(\text{ct}[8] \oplus K_{02}) \oplus \text{ISB}(\text{ct}_f[8] \oplus K_{02}) \\
    \text{ISB}(\text{ct}[5] \oplus K_{11}) \oplus \text{ISB}(\text{ct}_f[5] \oplus K_{11}) &= \text{mul3}\cdot \left( \text{ISB}(\text{ct}[8] \oplus K_{02}) \oplus \text{ISB}(\text{ct}_f[8] \oplus K_{02}) \right)
    \end{align*}
Similarly for keybytes $K_{03}$,$K_{10}$,$K_{21}$ and $K_{30}$ we can frame the following equations:
\begin{align*}
    \text{ISB}(\text{ct}[9] \oplus K_{12}) \oplus \text{ISB}(\text{ct}_f[9] \oplus K_{12}) &= \text{mul2} \cdot \left( \text{ISB}(\text{ct}[6] \oplus K_{21}) \oplus \text{ISB}(\text{ct}_f[6] \oplus K_{21}) \right) \\
    \text{ISB}(\text{ct}[3] \oplus K_{30}) \oplus \text{ISB}(\text{ct}_f[3] \oplus K_{30}) &= \text{ISB}(\text{ct}[6] \oplus K_{21}) \oplus \text{ISB}(\text{ct}_f[6] \oplus K_{21}) \\
    \text{ISB}(\text{ct}[12] \oplus K_{03}) \oplus \text{ISB}(\text{ct}_f[12] \oplus K_{03}) &= \text{mul2} \cdot \left( \text{ISB}(\text{ct}[6] \oplus K_{21}) \oplus \text{ISB}(\text{ct}_f[6] \oplus K_{21}) \right)
    \end{align*}
Here in AES, \texttt{mul2} and \texttt{mul3} denote multiplication by 2 and 3 in the finite field $\mathrm{GF}(2^8)$, used in the MixColumns step to achieve diffusion through field arithmetic. Solving these 12 equations allows us to recover the key bytes of 10th Round and thus the complete key of AES-128.
\subsection{Using Voltage Glitching}
Firstly we have identified a probable location(5090 to 5120 samples) of the starting of 8th round from the power trace of the AES-128 running in the integrated target of CWLite-ARM. Then we have set the parametsers for voltage glitching as follows:
\begin{verbatim}
    scope.glitch.clk_src = "clkgen"
    scope.glitch.output = "glitch_only" 
    scope.glitch.trigger_src = "ext_single"
    scope.io.glitch_lp = True
    scope.io.glitch_hp = True
\end{verbatim}
\subsubsection{Result}
A glitch was introduced at location 5100 samples, which corresponds to the start of Round 8. The parametsers 
\begin{verbatim}
    scope.glitch.offset= -37.890625 
    scope.glitch.width= 37.109375
\end{verbatim}
were responsible for this fault injection.

\subsubsection{Analysis and Fault Propagation}
We have one correct ciphertext and one faulty ciphertext after the fault injection. As the key was already known to us we have compared the two ciphertexts to analyze the fault propagation using the AES-128 Decryption method. The following tables show the state of the AES matrix at various stages of the encryption process, both with and without the fault.


\subsubsection{After The Fault Injection At Starting of $8^{\text{th}}$ Round :}
\[
\begin{array}{ccc}
\textbf{Without Fault} & \textbf{With Fault } & \textbf{Difference } \\[1ex]
\begin{array}{|c|c|c|c|}
\hline
\texttt{96} & \texttt{e9} & \texttt{e9} & \texttt{3c} \\
\hline
\texttt{71} & \texttt{87} & \texttt{61} & \texttt{89} \\
\hline
\texttt{6a} & \texttt{91} & \texttt{04} & \texttt{13} \\
\hline
\texttt{e4} & \texttt{c7} & \texttt{90} & \texttt{ff} \\
\hline
\end{array} 
&

\begin{array}{|c|c|c|c|}
    \hline
    \texttt{96} & \texttt{e9} & \texttt{e9} & \texttt{52} \\
    \hline
    \texttt{52} & \texttt{87} & \texttt{61} & \texttt{89} \\
    \hline
    \texttt{6a} & \texttt{91} & \texttt{04} & \texttt{13} \\
    \hline
    \texttt{e4} & \texttt{c7} & \texttt{90} & \texttt{ff} \\
    \hline
    \end{array}

    &

\begin{array}{|c|c|c|c|}
    \hline
    \texttt{00} & \texttt{00} & \texttt{00} & \fcolorbox{orange!80!black}{orange!20}{\texttt{6e}} \\
    \hline
    \fcolorbox{orange!80!black}{orange!20}{\texttt{23}} & \texttt{00} & \texttt{00} & \texttt{00} \\
    \hline
    \texttt{00} & \texttt{00} & \texttt{00} & \texttt{00} \\
    \hline
    \texttt{00} & \texttt{00} & \texttt{00} & \texttt{00} \\
    \hline
    \end{array}
\end{array}
\]

\subsubsection{After The Completion of $8^{\text{th}}$ Round(SubByte,ShiftRow,MixColumn,AddRoundkey(8)) :}
\[
\begin{array}{ccc}
\textbf{Without Fault} & \textbf{With Fault } & \textbf{Difference } \\[1ex]
\begin{array}{|c|c|c|c|}
\hline
\texttt{0c} & \texttt{b7} & \texttt{3b} & \texttt{ad} \\
\hline
\texttt{77} & \texttt{b8} & \texttt{a0} & \texttt{c3} \\
\hline
\texttt{31} & \texttt{0a} & \texttt{19} & \texttt{d8} \\
\hline
\texttt{43} & \texttt{b0} & \texttt{70} & \texttt{eb} \\
\hline
\end{array} 
&

\begin{array}{|c|c|c|c|}
    \hline
\texttt{0c} & \texttt{b7} & \texttt{3b} & \texttt{9e} \\
\hline
\texttt{77} & \texttt{b8} & \texttt{a0} & \texttt{75} \\
\hline
\texttt{31} & \texttt{0a} & \texttt{19} & \texttt{90} \\
\hline
\texttt{43} & \texttt{b0} & \texttt{70} & \texttt{6e} \\
\hline
    \end{array}

    &

\begin{array}{|c|c|c|c|}
    \hline
    \texttt{00} & \texttt{00} & \texttt{00} & \fcolorbox{orange!80!black}{orange!20}{\texttt{33}} \\
    \hline
    \texttt{00} & \texttt{00} & \texttt{00} & \fcolorbox{orange!80!black}{orange!20}{\texttt{b6}} \\
    \hline
    \texttt{00} & \texttt{00} & \texttt{00} & \fcolorbox{orange!80!black}{orange!20}{\texttt{48}} \\
    \hline
    \texttt{00} & \texttt{00} & \texttt{00} & \fcolorbox{orange!80!black}{orange!20}{\texttt{85}} \\
    \hline
    \end{array}
\end{array}
\]

\subsubsection{After The Completion of $9^{\text{th}}$ Round(SubByte,ShiftRow,MixColumn,AddRoundkey(9)) :}
\[
\begin{array}{ccc}
\textbf{Without Fault} & \textbf{With Fault } & \textbf{Difference } \\[1ex]
\begin{array}{|c|c|c|c|}
    \hline
    \texttt{c2} & \texttt{10} & \texttt{a5} & \texttt{54} \\
    \hline
    \texttt{df} & \texttt{31} & \texttt{da} & \texttt{c0} \\
    \hline
    \texttt{67} & \texttt{79} & \texttt{42} & \texttt{5d} \\
    \hline
    \texttt{9b} & \texttt{74} & \texttt{40} & \texttt{fa} \\
    \hline
\end{array} 
&

\begin{array}{|c|c|c|c|}
    \hline
    \texttt{b4} & \texttt{11} & \texttt{6b} & \texttt{73} \\
    \hline
    \texttt{a9} & \texttt{32} & \texttt{a7} & \texttt{5e} \\
    \hline
    \texttt{fd} & \texttt{7b} & \texttt{f1} & \texttt{c3} \\
    \hline
    \texttt{77} & \texttt{75} & \texttt{f3} & \texttt{43} \\
    \hline
    \end{array}

    &

    \begin{array}{|c|c|c|c|}
        \hline
        \fcolorbox{orange!80!black}{orange!20}{\texttt{76}} & \fcolorbox{orange!80!black}{orange!20}{\texttt{01}} & \fcolorbox{orange!80!black}{orange!20}{\texttt{ce}} & \fcolorbox{orange!80!black}{orange!20}{\texttt{27}} \\
        \hline
        \fcolorbox{orange!80!black}{orange!20}{\texttt{76}} & \fcolorbox{orange!80!black}{orange!20}{\texttt{03}} & \fcolorbox{orange!80!black}{orange!20}{\texttt{7d}} & \fcolorbox{orange!80!black}{orange!20}{\texttt{9e}} \\
        \hline
        \fcolorbox{orange!80!black}{orange!20}{\texttt{9a}} & \fcolorbox{orange!80!black}{orange!20}{\texttt{02}} & \fcolorbox{orange!80!black}{orange!20}{\texttt{b3}} & \fcolorbox{orange!80!black}{orange!20}{\texttt{9e}} \\
        \hline
        \fcolorbox{orange!80!black}{orange!20}{\texttt{ec}} & \fcolorbox{orange!80!black}{orange!20}{\texttt{01}} & \fcolorbox{orange!80!black}{orange!20}{\texttt{b3}} & \fcolorbox{orange!80!black}{orange!20}{\texttt{b9}} \\
        \hline
    \end{array}
\end{array}
\]
\subsubsection{After The Completion of $10^{\text{th}}$ Round(SubByte,ShiftRow,AddRoundkey(10)) :}
\[
\begin{array}{ccc}
\textbf{Without Fault} & \textbf{With Fault } & \textbf{Difference } \\[1ex]
\begin{array}{|c|c|c|c|}
    \hline
    \texttt{f5} & \texttt{03} & \texttt{e7} & \texttt{96} \\
    \hline
    \texttt{d3} & \texttt{b9} & \texttt{85} & \texttt{fd} \\
    \hline
    \texttt{d5} & \texttt{69} & \texttt{89} & \texttt{ba} \\
    \hline
    \texttt{85} & \texttt{9d} & \texttt{5a} & \texttt{af} \\
    \hline
\end{array} 
&

\begin{array}{|c|c|c|c|}
    \hline
    \texttt{5d} & \texttt{4b} & \texttt{9e} & \texttt{39} \\
    \hline
    \texttt{37} & \texttt{b2} & \texttt{67} & \texttt{b0} \\
    \hline
    \texttt{58} & \texttt{0b} & \texttt{58} & \texttt{2d} \\
    \hline
    \texttt{b2} & \texttt{7c} & \texttt{55} & \texttt{ab} \\
    \hline
    \end{array}

    &

\begin{array}{|c|c|c|c|}
    \hline
    \fcolorbox{orange!80!black}{orange!20}{\texttt{a8}} & \fcolorbox{orange!80!black}{orange!20}{\texttt{48}} & \fcolorbox{orange!80!black}{orange!20}{\texttt{79}} & \fcolorbox{orange!80!black}{orange!20}{\texttt{af}} \\
    \hline
    \fcolorbox{orange!80!black}{orange!20}{\texttt{e4}} & \fcolorbox{orange!80!black}{orange!20}{\texttt{0b}} & \fcolorbox{orange!80!black}{orange!20}{\texttt{e2}} & \fcolorbox{orange!80!black}{orange!20}{\texttt{4d}} \\
    \hline
    \fcolorbox{orange!80!black}{orange!20}{\texttt{8d}} & \fcolorbox{orange!80!black}{orange!20}{\texttt{62}} & \fcolorbox{orange!80!black}{orange!20}{\texttt{d1}} & \fcolorbox{orange!80!black}{orange!20}{\texttt{97}} \\
    \hline
    \fcolorbox{orange!80!black}{orange!20}{\texttt{37}} & \fcolorbox{orange!80!black}{orange!20}{\texttt{e1}} & \fcolorbox{orange!80!black}{orange!20}{\texttt{0f}} & \fcolorbox{orange!80!black}{orange!20}{\texttt{04}} \\
    \hline
    \end{array}
\end{array}
\]
Here we notice that the fault was injected at 2 bytes of the $D_3$ diagonal of the state matrix of 8th round. According to the fault model proposed by Dhiman et al.,\cite{Saha2009ADF} the fault in the $D_3$ diagonal of the state matrix at the start of Round 8 propagates through the subsequent rounds, affecting the final ciphertext.According to The Byte inter-relations corresponding to $D_3$ diagonal, we can express the following relation:
\[
\begin{array}{ccc}
    \textbf{After 8th Round} & \textbf{After 9th Round} & \textbf{After 10th Round Shift Row} \\[1ex]
    \begin{array}{|c|c|c|c|}
        \hline
        \texttt{ } & \texttt{ } &  \texttt{ } & \texttt{$f_1$}\\
        \hline
        \texttt{ } & \texttt{ } &  \texttt{ } & \texttt{$f_2$}  \\
        \hline
        \texttt{ } & \texttt{ } &  \texttt{ } & \texttt{$f_3$}  \\
        \hline
        \texttt{ } & \texttt{ } &  \texttt{ } & \texttt{$f_4$} \\
        \hline
    \end{array}
&

\begin{array}{|c|c|c|c|}
    \hline
    \texttt{$f_4$} & \texttt{$f_3$} & \texttt{3$f_2$} & \texttt{$2f_1$} \\
    \hline
    \texttt{$f_4$} & \texttt{3$f_3$} & \texttt{2$f_2$} & \texttt{$f_1$} \\
    \hline
    \texttt{3$f_4$} & \texttt{2$f_3$} & \texttt{$f_2$} & \texttt{$f_1$} \\
    \hline
    \texttt{2$f_4$} & \texttt{$f_3$} & \texttt{$f_2$} & \texttt{3$f_1$} \\
    \hline
\end{array}

    &

    \begin{array}{|c|c|c|c|}
        \hline
        \texttt{$f_4$} & \texttt{$f_3$} & \texttt{3$f_2$} & \texttt{$2f_1$} \\
        \hline
        \texttt{3$f_3$} & \texttt{2$f_2$} & \texttt{$f_1$} & \texttt{$f_4$} \\
        \hline
        \texttt{$f_2$} & \texttt{$f_1$} & \texttt{3$f_4$} & \texttt{2$f_3$} \\
        \hline
        \texttt{3$f_1$} & \texttt{2$f_4$} & \texttt{$f_3$} & \texttt{$f_2$}\\
        \hline
    \end{array}
    
\end{array}
\]

Again we can make the following equations based on the differences in the ciphertexts with or without the fault injection to guess the key bytes $K_{00}$,$K_{13}$,$K_{22}$ and $K_{31}$ . The equations are as follows:
\begin{align*}
    \text{ISB}(\text{ct}[0] \oplus K_{00}) \oplus \text{ISB}(\text{ct}_f[0] \oplus K_{00}) &= \text{ISB}(\text{ct}[13] \oplus K_{13}) \oplus \text{ISB}(\text{ct}_f[13] \oplus K_{13}) \\
    \text{ISB}(\text{ct}[10] \oplus K_{22}) \oplus \text{ISB}(\text{ct}_f[10] \oplus K_{22}) &= \text{mul3}\cdot \left( \text{ISB}(\text{ct}[13] \oplus K_{13}) \oplus \text{ISB}(\text{ct}_f[13] \oplus K_{13}) \right) \\
    \text{ISB}(\text{ct}[7] \oplus K_{31}) \oplus \text{ISB}(\text{ct}_f[7] \oplus K_{31}) &= \text{mul2} \cdot \left( \text{ISB}(\text{ct}[13] \oplus K_{13}) \oplus \text{ISB}(\text{ct}_f[13] \oplus K_{13}) \right)
    \end{align*}
Similarly for keybytes $K_{01}$,$K_{10}$,$K_{23}$ and $K_{32}$ we can frame the following equations:
\begin{align*}
    \text{ISB}(\text{ct}[11] \oplus K_{32}) \oplus \text{ISB}(\text{ct}_f[11] \oplus K_{32}) &= \text{ISB}(\text{ct}[4] \oplus K_{01}) \oplus \text{ISB}(\text{ct}_f[4] \oplus K_{01}) \\
    \text{ISB}(\text{ct}[1] \oplus K_{10}) \oplus \text{ISB}(\text{ct}_f[1] \oplus K_{10}) &= \texttt{mul3}\left( \text{ISB}(\text{ct}[11] \oplus K_{32}) \oplus \text{ISB}(\text{ct}_f[11] \oplus K_{32}) \right) \\
    \text{ISB}(\text{ct}[14] \oplus K_{23}) \oplus \text{ISB}(\text{ct}_f[14] \oplus K_{23}) &= \texttt{mul2}\left( \text{ISB}(\text{ct}[4] \oplus K_{01}) \oplus \text{ISB}(\text{ct}_f[4] \oplus K_{01}) \right)
    \end{align*}
For keybytes $K_{02}$,$K_{11}$,$K_{20}$ and $K_{33}$ we can frame the following equations:
\begin{align*}
    \text{ISB}(\text{ct}[8] \oplus K_{02}) \oplus \text{ISB}(\text{ct}_f[8] \oplus K_{02}) &= \texttt{mul3} \left( \text{ISB}(\text{ct}[2] \oplus K_{20}) \oplus \text{ISB}(\text{ct}_f[2] \oplus K_{20}) \right) \\
    \text{ISB}(\text{ct}[15] \oplus K_{33}) \oplus \text{ISB}(\text{ct}_f[15] \oplus K_{33}) &= \text{ISB}(\text{ct}[8] \oplus K_{02}) \oplus \text{ISB}(\text{ct}_f[8] \oplus K_{02}) \\
    \text{ISB}(\text{ct}[5] \oplus K_{11}) \oplus \text{ISB}(\text{ct}_f[5] \oplus K_{11}) &= \texttt{mul2} \left( \text{ISB}(\text{ct}[2] \oplus K_{20}) \oplus \text{ISB}(\text{ct}_f[2] \oplus K_{20}) \right)
    \end{align*}

For keybytes $K_{03}$,$K_{10}$,$K_{21}$ and $K_{30}$ we can frame the following equations:
\begin{align*}
    \text{ISB}(\text{ct}[9] \oplus K_{12}) \oplus \text{ISB}(\text{ct}_f[9] \oplus K_{12}) &= \text{ISB}(\text{ct}[6] \oplus K_{21}) \oplus \text{ISB}(\text{ct}_f[6] \oplus K_{21}) \\
    \text{ISB}(\text{ct}[3] \oplus K_{30}) \oplus \text{ISB}(\text{ct}_f[3] \oplus K_{30}) &= \texttt{mul3}\left( \text{ISB}(\text{ct}[6] \oplus K_{21}) \oplus \text{ISB}(\text{ct}_f[6] \oplus K_{21}) \right) \\
    \text{ISB}(\text{ct}[12] \oplus K_{03}) \oplus \text{ISB}(\text{ct}_f[12] \oplus K_{03}) &= \texttt{mul2}\left( \text{ISB}(\text{ct}[6] \oplus K_{21}) \oplus \text{ISB}(\text{ct}_f[6] \oplus K_{21}) \right)
    \end{align*}

\section{Conclusion}
In this experiment, we successfully demonstrated the use of CWLite for voltage glitching and Clock glitching to inject faults into the AES-128 encryption process. By analyzing the fault propagation through the rounds of AES, we were able to derive equations that allowed us to recover key bytes from the faulty ciphertext as mentioned in \cite{Saha2009ADF}. These attacks can also be perfomed with the help of CWHusky with slight modification in the setting by making 
\begin{verbatim}
    scope.glitch.clk_src = "pll"
\end{verbatim} 
and then finding a suitable glitch location observing Power Trace, width and offset.
We have observed few glitches in CWHusky as follows:
\begin{table}[h!]
    \centering
    \begin{tabular}{|l|c|c|c|}
    \hline
    \textbf{Glitch Location} & \textbf{Width} & \textbf{Offset} & \textbf{Effected Diagonal} \\
    \hline
    % Add your data rows below
    6123 & 50 & 10 & $D_3$ \\
    6134 & 100 & 10 & $D_2$,$D_3$ \\
    6134 & 500 & 10 & $D_2$,$D_3$ \\
    6129 & 50 & 50 & $D_2$ \\
    \hline
    \end{tabular}
    \caption{Table showing location, width, offset, and whether the diagonal is affected}
    \label{tab:location_data}
    \end{table}
Similar to the previous section, we can analyze the fault propagation and recover the key bytes by framing the equations from the faulty ciphertext and correct ciphertext.
These results demonstrate the effectiveness of CWLite and CWHusky in performing fault injection attacks on AES-128 encryption. The ability to manipulate the encryption process through voltage and clock glitches provides valuable insights into the security vulnerabilities of cryptographic systems.
 

