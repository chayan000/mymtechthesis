Fault injection attacks are considered as a potent threat to embedded systems that exploites hardware-level vulnerabilities through techniques such as Voltage or Clock glitching. These attacks can compromise cryptographic implementations by inducing faults that alter a devices normal execution flow. This section explores the major studies and developments in the field.

\subsection*{Non-Invasive Fault Injection Techniques}

comprehensive survey of non-invasive fault injection techniques, such as voltage glitching, clock manipulation, and electromagnetic interference is provided by Mazumder et al.~\cite{mazumder2023comprehensive} . These approaches enable attackers to disrupt a device's operation without needing direct physical access or modifications, making them especially practical and dangerous in real-world attack scenarios.

\subsection*{Practicality of FIAs}
Breier and Hou~\cite{breier2022practical} examine the practicality of executing fault attacks across various architectures. Their work demonstrates that low-cost setups can achieve successful fault injections on widely used processors, raising concerns about the accessibility of such attacks.

\subsection*{Software-Level Implications}
Yuce et al.\cite{yuce2020fault} focus on how FIAs affect embedded software systems. They detail how instruction and data flows can be corrupted, potentially bypassing authentication or exposing sensitive information.



\subsection*{Evaluation of Fault Injection Tools}
Brito et al. \cite{brito2023evaluation} conduct a comparative evaluation of fault injection platforms, helping researchers select suitable tools for reliability testing of embedded systems under attack conditions.
\subsection*{Fault Injection Attacks on Cryptographic Devices: Theory, Practice, and Countermeasures}

Barenghi et al. \cite{6178001} discuss how FIAs exploit hardware imperfections by introducing faults—via voltage glitches, clock manipulation, or laser/electromagnetic interference—to compromise cryptographic computations. These attacks are classified based on cost and complexity: \textbf{low-cost attacks} are accessible with modest equipment, whereas \textbf{high-cost attacks} require specialized tools and expertise.

Research has demonstrated practical FIAs against major ciphers, often leading to key recovery or algorithm compromise. Countermeasures include hardware-based fault detection, redundancy in computation, and intrusion monitoring systems. Additionally, studies show the combined use of fault and power analysis attacks can further undermine device security.

The literature emphasizes the need for robust, layered defenses that balance performance, cost, and security in embedded cryptographic implementations.

\subsection*{Application to Post-Quantum Cryptography}
Hermelink et al. \cite{hermelink2023side} investigate the resilience of post-quantum algorithms, particularly Kyber and Dilithium, against both side-channel and fault injection attacks. This research is crucial as the cryptographic landscape shifts towards quantum-resistant designs.


\subsection*{Glitching Attacks on Post-Quantum Cryptography: A Focus on Kyber}

In recent years, the resilience of lattice-based schemes like Kyber against physical attacks has become an active area of research. Ravi et al.~\cite{ravi2021fault} explored voltage and clock glitching as a means to compromise Kyber implementations. Their study demonstrated that Kyber is not inherently resistant to low-level fault injection techniques and identified several fault models that can lead to secret key recovery. Interestingly, the authors showed that inducing faults during the decryption phase can leak sufficient information through erroneous outputs, allowing attackers to infer sensitive intermediate values. The work emphasizes that even theoretically secure schemes require robust physical implementations to remain secure in real-world devices.

