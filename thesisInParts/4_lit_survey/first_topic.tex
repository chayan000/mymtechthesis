Fault injection attacks are considered as a potent threat to embedded systems that exploites hardware-level vulnerabilities through techniques such as Voltage or Clock glitching. These attacks can compromise cryptographic implementations by inducing faults that alter a devices normal execution flow. This section explores the major studies and developments in the field.

\subsection*{Non-Invasive Fault Injection Techniques \cite{mazumder2023comprehensive}}

The work by Mazumder et al. provides a detailed and structured overview of several fault injection methods that do not require physical tampering or invasive modification of the target device. These techniques, including voltage glitching, clock manipulation, and electromagnetic interference (EMI), allow attackers to induce faults in embedded systems in a stealthy and highly practical manner. Voltage glitching involves introducing brief, controlled power supply interruptions that can bypass security checks or cause unpredictable behavior in microcontrollers. Clock manipulation works similarly by introducing timing errors through changes to the device’s clock signal, often leading to skipped instructions or corrupted computations. Electromagnetic interference, on the other hand, uses precisely timed EM pulses to disrupt the device's logic circuits, sometimes from a distance, making it even more covert.

What makes these methods particularly concerning is their non-invasive nature they do not require the chip to be opened, nor do they leave obvious physical traces. This makes them attractive options for attackers, especially in scenarios where direct physical access is limited or where tamper detection mechanisms are in place. The paper highlights not only the technical mechanisms behind each technique but also their application in real-world attack cases, such as bypassing authentication, altering cryptographic operations, or enabling privilege escalation.

Furthermore, Mazumder et al. emphasize how these attacks are becoming more accessible due to the availability of low-cost tools (like the ChipWhisperer platform), thereby increasing their threat potential. The survey also discusses the challenges in detecting and mitigating such attacks, pointing out that many traditional countermeasures are insufficient against fault models introduced non-invasively. This work serves as both a warning and a guide, showing that understanding and preparing for non-invasive faults is no longer optional but a necessity in secure hardware design.



\subsection*{Practicality of Fault Injection Attacks \cite{breier2022practical}}
Breier and Hou’s study delves into the real-world feasibility of performing fault injection attacks across different processor architectures, including both low-end microcontrollers and more complex embedded CPUs. The core message of their work is that fault attacks are no longer limited to well-funded laboratories or specialized equipment rather, they are increasingly practical using inexpensive, off-the-shelf components. The authors explore various fault injection methods, such as clock glitching and electromagnetic pulse attacks, and show how these can be tuned to reliably disrupt the execution of critical instructions in real devices. Their experiments reveal that even widely deployed platforms, including ARM Cortex-M and AVR microcontrollers, are vulnerable to such low-cost fault attacks. By systematically analyzing fault models, setup complexity, success rates, and reproducibility, they establish that minimal hardware knowledge and modest resources are enough to mount powerful attacks on systems that were previously considered secure. This raises significant alarms for device manufacturers and designers, especially in domains like IoT and automotive systems, where security is crucial but physical access to devices is often possible. Breier and Hou argue for stronger built-in defenses at both hardware and firmware levels to mitigate these growing risks, highlighting the urgency of fault-aware system design in today's threat landscape.

\subsection*{Software-Level Implications \cite{yuce2020fault}}
In their work Yuce et al. explore how fault injection attacks (FIAs) extend beyond hardware disruption and pose serious threats to the integrity of embedded software systems. The paper emphasizes that even transient hardware faults, when timed precisely, can alter the behavior of executing software in subtle but dangerous ways. Their analysis covers how faults can corrupt instruction fetches, alter control flow, or tamper with data variables during critical operations. These disruptions can lead to skipped authentication steps, altered logical decisions, or exposure of cryptographic secrets in memory. One of the key insights from the paper is that many embedded applications do not include robust error detection mechanisms at the software level, making them especially vulnerable to attacks like instruction skips or conditional branch manipulation. The authors also discuss how certain software constructs, such as loops, conditional branches, and exception handling, are more prone to fault exploitation. Their findings suggest that secure software development must go hand-in-hand with hardware protections, advocating for software-based countermeasures such as redundant computations, control flow integrity checks, and fault detection routines. Overall, Yuce et al. demonstrate that software is not just a passive victim in fault attacks it is often the weakest link, and without deliberate protection strategies, it can be easily subverted through precise fault injection.


\subsection*{Evaluation of Fault Injection Tools \cite{brito2023evaluation}}
Brito et al. present a thorough comparative study of fault injection platforms, aiming to assist researchers and practitioners in selecting the most appropriate tools for evaluating the resilience of embedded systems against fault based attacks. Their work systematically benchmarks various open-source and commercial fault injection solutions, focusing on factors such as ease of use, cost, precision, supported fault models, and compatibility with different hardware targets. The evaluation covers widely used platforms like ChipWhisperer, EMFI rigs, and voltage/clock glitching setups, highlighting their relative strengths and limitations. For example, they show how some platforms excel at fine-grained glitch timing but may lack in automation or scalability, while others offer broader integration with software testing frameworks. Importantly, Brito et al. emphasize the trade-offs between accuracy and accessibility while high-end tools offer better fault localization, low-cost setups still provide sufficient capability for meaningful attack simulations. The study concludes by identifying gaps in current tools, such as limited support for post quantum cryptography evaluation and the lack of standardized metrics for cross platform comparison. Their work contributes valuable insights to the field, serving not only as a buyer's guide for fault injection equipment but also as a roadmap for future tool development tailored to modern embedded security challenges.

\subsection*{Introduction to ChipWhisperer \cite{flynnpaper}}
The paper “ChipWhisperer: An Open-Source Platform for Hardware Embedded Security Research” by Colin O’Flynn and Zhizhang (David) Chen presents one of the first comprehensive, affordable, and openly available solutions for performing side-channel power analysis and fault injection attacks on embedded systems. The authors recognized that many academic and practical advances in hardware security were held back by the lack of accessible tools and proposed ChipWhisperer as a unified platform combining hardware and software to bridge this gap. It integrates key components such as a synchronous sampling oscilloscope, programmable glitch generator, and target microcontrollers into a single cohesive system that supports real-time experiments. The paper details the system’s architecture, design considerations, and validation through successful implementation of side-channel attacks, like differential power analysis (DPA) on AES. Unlike previous expensive commercial equipment, ChipWhisperer allows researchers and educators to conduct advanced attacks at low cost with high precision. This work has significantly lowered the barrier for entry in embedded security research, enabled reproducible experimentation, and sparked a wave of open-hardware initiatives in the field. It continues to be a foundational tool in academia and industry for teaching, research, and development in hardware cryptanalysis and countermeasure testing.


\subsection*{Fault Injection Attacks on Cryptographic Devices: Theory, Practice, and Countermeasures \cite{6178001}}
Barenghi et al. have discussed how FIAs exploit hardware imperfections by introducing faults via voltage glitches, clock manipulation, or laser/electromagnetic interference—to compromise cryptographic computations. These attacks are classified based on cost and complexity: low-cost attacks are accessible with modest equipment, whereas high-cost attacks require specialized tools and expertise.

Their research has demonstrated practical FIAs against major ciphers, often leading to key recovery or algorithm compromise. Countermeasures include hardware-based fault detection, redundancy in computation, and intrusion monitoring systems. Additionally, studies show the combined use of fault and power analysis attacks can further undermine device security.

The literature emphasizes the need for robust, layered defenses that balance performance, cost, and security in embedded cryptographic implementations.

\subsection*{Application to Post-Quantum Cryptography \cite{hermelink2023side}}
Hermelink et al. investigate the vulnerability of post-quantum cryptographic (PQC) schemes specifically Kyber and Dilithium to both side channel and fault injection attacks, at a time when these algorithms are being adopted as part of the NIST standardization process for quantum safe security. The study explores how classical attack vectors like voltage glitching and differential fault analysis can still threaten the integrity of lattice based schemes, even though they are mathematically secure against quantum computing. By targeting specific stages in the key generation, encryption, or signature routines, the researchers demonstrate that well timed faults can lead to key leakage, faulty outputs, or invalid signature verifications. Their experiments involve both simulated fault environments and practical hardware setups, providing a well rounded assessment of how real world systems might be exploited. A notable contribution of their work is the identification of fault sensitive areas in PQC implementations that often lack robust error handling or detection, particularly under embedded resource constraints. This highlights the urgent need for designing PQC implementations with built in fault tolerance, secure hardware integration, and comprehensive testing under physical attack scenarios. As PQC adoption accelerates, Hermelink et al.'s work serves as a critical reminder that resistance to mathematical attacks must be complemented with resilience to implementation-level threats.

\subsection*{Diagonal Fault Injection Attack on AES-128 \cite{Saha2009ADF}}
The paper \textit{``A Diagonal Fault Attack on the Advanced Encryption Standard''} (ePrint 2009/581) by Saha et al.\ introduces a novel and practical fault attack technique on AES-128 that targets specific diagonals in the AES state matrix during the 8th encryption round. Unlike earlier random or byte-level fault models, this approach exploits the internal structure of AES particularly the effects of \texttt{ShiftRows} and \texttt{MixColumns} by injecting a fault into one diagonal, thereby enabling partial key recovery with reduced computational effort. Demonstrated on an FPGA implementation, the attack could recover the full key with just one well-placed fault and moderate brute-force computation (approximately $2^{32}$). The method proved significantly more efficient than traditional models, highlighting the need for fault-aware hardware security designs. This structured fault model has influenced later works and emphasized the importance of designing countermeasures against position-specific and low-overhead fault injection attacks.


\subsection*{Glitching Attacks on Post-Quantum Cryptography: A Focus on Kyber ~\cite{ravi2021fault}}

In recent years, the resilience of lattice-based schemes like Kyber against physical attacks has become an active area of research. Ravi et al. explored voltage and clock glitching as a means to compromise Kyber implementations. Their study demonstrated that Kyber is not inherently resistant to low-level fault injection techniques and identified several fault models that can lead to secret key recovery. Interestingly, the authors showed that inducing faults during the decryption phase can leak sufficient information through erroneous outputs, allowing attackers to infer sensitive intermediate values. The work emphasizes that even theoretically secure schemes require robust physical implementations to remain secure in real-world devices.

\subsection*{Fault Injection Analysis of the NTT in Kyber and Dilithium \cite{Ravi_Yang_Bhasin_Zhang_Chattopadhyay_2023} }
Ravi et al. presented a pioneering analysis of the vulnerability of the Number Theoretic Transform (NTT) to fault injection attacks. As a fundamental component in structured lattice based cryptographic systems particularly in key encapsulation mechanisms (KEMs) and digital signatures, the NTT plays a critical role in secure polynomial multiplication. The authors uncovered a previously unreported weakness: a single, strategically induced fault in the NTT drastically reduces the output’s entropy. Exploiting this flaw, they introduced a suite of attacks, including key and message recovery attacks on the Kyber KEM during both key generation and encryption phases. Additionally, they developed innovative existential forgery attacks against both deterministic and randomized signing processes in the Dilithium signature scheme, along with a novel method to bypass signature verification altogether. These attacks are validated through electromagnetic fault injection on real hardware, using highly optimized Kyber and Dilithium implementations from the pqm4 library running on an ARM Cortex-M4 microcontroller, demonstrating consistently high success rates.