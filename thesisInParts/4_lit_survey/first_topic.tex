Fault injection attacks (FIAs) have emerged as a potent threat to embedded systems, exploiting hardware-level vulnerabilities through techniques such as voltage and clock glitching. These attacks can compromise cryptographic implementations by inducing faults that alter a devices normal execution flow. This section explores the major studies and developments in the field.

\subsection{Non-Invasive Fault Injection Techniques}
Mazumder et al.~\cite{mazumder2023comprehensive} present a detailed survey of non-invasive fault injection methods, including voltage, clock, and electromagnetic techniques. These methods allow adversaries to disrupt device operations without direct physical tampering, making them highly relevant in real-world attack scenarios.

\subsection{Practicality of FIAs}
Breier and Hou~\cite{breier2022practical} examine the practicality of executing fault attacks across various architectures. Their work demonstrates that low-cost setups can achieve successful fault injections on widely used processors, raising concerns about the accessibility of such attacks.

\subsection{Software-Level Implications}
Yuce et al.\cite{yuce2020fault} focus on how FIAs affect embedded software systems. They detail how instruction and data flows can be corrupted, potentially bypassing authentication or exposing sensitive information.



\subsection{Evaluation of Fault Injection Tools}
Brito et al. \cite{brito2023evaluation} conduct a comparative evaluation of fault injection platforms, helping researchers select suitable tools for reliability testing of embedded systems under attack conditions.

\subsection{Application to Post-Quantum Cryptography}
Hermelink et al. \cite{hermelink2023side} investigate the resilience of post-quantum algorithms, particularly Kyber and Dilithium, against both side-channel and fault injection attacks. This research is crucial as the cryptographic landscape shifts towards quantum-resistant designs.
