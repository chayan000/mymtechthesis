

\section{Introduction}
The global cryptographic community is facing a growing concern: the rise of quantum computing is threatning the security foundations of many widely used cryptographic protocols. Algorithms such as RSA, DSA, and ECC, are susceptible to quantum attacks. Most notably Shor’s algorithm, which can factor large integers and compute discrete logarithms exponentially faster than classical algorithms. This looming threat has spurred the development of Post-Quantum Cryptography (PQC), which refers to cryptographic schemes believed to be resistant to attacks by quantum computers.

PQC does not depend on the number-theoretic problems that quantum computers can efficiently solve. Instead, it builds on hard mathematical problems such as lattice-based, hash-based, code-based, and multivariate polynomial problems. Among these, lattice-based cryptography has emerged as a strong candidate due to its balance between security, efficiency, and versatility. One of the most promising lattice-based schemes is Kyber, a key encapsulation mechanism (KEM) that is part of the  PQC standardization process of NIST. Kyber is based on the Module Learning With Errors (MLWE) problem, which is widely considered secure even in the quantum setting.

The transition to PQC is not only a matter of replacing algorithms but also ensuring their robustness against a wide range of implementation attacks, including fault analysis. While PQC schemes may be theoretically secure, their implementations can still be vulnerable to side-channel and fault injection attacks—forms of active attacks where adversaries induce errors during computation to extract secret information. This report focuses on preparing a Kyber implementation to withstand fault analysis, exploring its vulnerabilities, and proposing countermeasures to secure its deployment in the post-quantum era.

\section{Overview of Kyber}
Kyber is a lattice-based key encapsulation mechanism (KEM) designed to provide secure key exchange in a post-quantum world. It is built on the Module Learning With Errors (MLWE) problem, which is believed to be hard even for quantum computers. Kyber offers several advantages, including efficient performance, small key sizes, and strong security guarantees.
Kyber operates by encapsulating a symmetric key within a public key, allowing two parties to securely exchange keys without directly sharing them. The encapsulation process involves generating a random polynomial, encoding it with an error term, and then encrypting it using the recipient's public key. The recipient can then decrypt the encapsulated key using their private key, ensuring that only they can access the shared secret.
\begin{table}[h!]
    \centering
    \begin{tabularx}{\textwidth}{l *{4}{>{\centering\arraybackslash}X}}
      \toprule
      \textbf{Variant} & \textbf{Security Level (bits)} & \textbf{Public Key (bytes)} & \textbf{Private Key (bytes)} & \textbf{Ciphertext (bytes)} \\
      \midrule
      Kyber512  & 128 & 800   & 1,632 & 768   \\
      Kyber768  & 192 & 1,184 & 2,400 & 1,088 \\
      Kyber1024 & 256 & 1,568 & 3,168 & 1,568 \\
      \bottomrule
    \end{tabularx}
    \caption{Three Variants of Kyber Key Encapsulation Mechanism}
    \label{tab:kyber-parameters}
    \end{table}
    
The table above summarizes the key parameters of the three variants of the Kyber key encapsulation mechanism (KEM). Each variant is designed to provide different levels of security while balancing performance and resource requirements. 
Kyber512, Kyber768, and Kyber1024, each variant corresponds to a specific security level and is optimized for different use cases.

Overall, the table provides a clear comparison of the different Kyber variants, highlighting their trade-offs between security, performance, and resource utilization, which are critical considerations in the design of post-quantum cryptographic systems.
The diagram \ref{fig:kyber_overview} taken from \cite{10.1145/3603170} illustrates the high-level structure of Kyber's key encapsulation mechanism, showing the encapsulation and decapsulation processes. The security of Kyber relies on the hardness of the underlying MLWE problem, making it a strong candidate for post-quantum cryptography.
\begin{figure}[h]
    \centering
    \includegraphics[width=0.9\textwidth]{/home/cp07/mymtechthesis/thesisInParts/images/kyber_kem.png}
    \caption{Overview of Kyber's Key Encapsulation Mechanism}
    \label{fig:kyber_overview}
    

\end{figure}

\section{Fault Injection in Kyber using ChipWhisperer}
The Kyber implementation utilized in this work originates from the official reference code submitted to the National Institute of Standards and Technology (NIST) as part of their Post-Quantum Cryptography (PQC) Standardization Project Round 3 \cite{nistpqc2020}. The reference implementation is written in portable C and primarily designed for algorithm validation and benchmarking on conventional computing platforms.

To adapt the Kyber implementation for hardware-based fault analysis, the codebase was modified to support the SimpleSerial v2 protocol. SimpleSerial provides a lightweight and deterministic serial communication interface, commonly used with side-channel and fault injection tools like ChipWhisperer. Integration involved encapsulating key Kyber operations, including \texttt{crypto\_kem
\_keypair}, \texttt{crypto\_kem\_enc}, and \texttt{crypto\_kem\_dec}, within SimpleSerial command handlers. This modification enables consistent command-response interaction and reliable triggering for precise measurement and injection. As a result, we were able to introduce fault in the keypair generation operation. However this was not a precise and useful attack to retrive any data but it sets the stage for further fault injection attacks on post quantum algorithms like Kyber.

\section{Challenges faced in preparing the attack}
The rng.c file present in Kyber implementation provides a pseudo-random byte generator,
In various implementations of Kyber, the method of generating random numbers varies based on the platform and performance goals. 

The official Round 3 reference implementation utilizes AES256 in counter mode (AES-CTR) as a deterministic random byte generator, adhering to the NIST standard (SP 800-90A). This implementation is written in portable C, prioritizing correctness and clarity.

For constrained devices, such as microcontrollers, the pqm4 version opts for a different approach by avoiding AES altogether. Instead, it uses SHAKE-based randomness, which is lighter and easier to implement on low-power chips that lack hardware AES support.
In the case of our implementation we have used the tinyaes128 for that purpose.tinyaes128 is implemented in ChipWhisperer as a C-based AES-128 function in the simpleserial-aes firmware project, tightly integrated with serial communication to allow real-time encryption and trace capture.

Another Problem arises when we try to see the public key (pk) or secret key (sk) using the $simpleseerial\_put$ method, which is used to send data over the serial interface. The issue is that the public key (pk) and secret key (sk) are large arrays, and the $simpleseerial\_put$ method can send only 256 bytes at a time. This limitation means that we cannot directly send the entire pk or sk in one go, as they exceed this byte limit. We have to send them in chunks, which requires careful management of the data to ensure that the entire key is transmitted correctly without losing any information. 

\section{Conclusion}
In this chapter, we have explored the challenges and considerations involved in preparing a Kyber implementation for fault analysis. By integrating the Kyber reference code with the SimpleSerial protocol and addressing the limitations of random number generation and data transmission, we have laid the groundwork for future fault injection attacks on Kyber. This work highlights the importance of robust implementation practices in ensuring the security of post-quantum cryptographic systems against both theoretical and practical threats.