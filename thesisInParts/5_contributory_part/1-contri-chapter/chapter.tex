 % Chanper text
 \section{Introduction}
 Fault injection attacks are getting a lot of attention in hardware security because they can quietly mess up embedded systems. Attackers sneak in faults at just the right moment, especially during important tasks like encryption, to find hidden weaknesses. As researchers dig deeper into these attacks, having a solid, easy-to-use, and well-explained lab setup is super important. A good lab helps get steady, repeatable results and makes it easier to build and test strong defenses for hardware.
 
 ChipWhisperer \cite{chipwhisperer2025} supports a range of techniques for fault injection, most notably Clock glitching and Voltage glitching fault injection. These methods can induce controlled errors in the behavior of a target device, often revealing vulnerabilities in cryptographic implementations or exposing unexpected execution paths. The platform’s ability to fine-tune fault parameters, such as glitch width, offset, and repetition, is essential for conducting systematic attacks and for understanding how and when systems fail under abnormal conditions. There are different types of ChipWhisperer hardwares available, the setup of CWLITE and CWHusky is shown bellow.

 ChipSHOUTER(CW520) \cite{chipshouter_doc} is a specialized tool used for electromagnetic (EM) fault injection attacks. It emits controlled EM pulses to simulate potential real world threats, helping researchers and engineers identify vulnerabilities. The installation process of ChipWhisperer is based on chipwhisperer and the process of seting up the CW520 can be found in \cite{chipshouter_manual_v1.4}. This comes with one pre programmed target and one programmable Target \cite{chipshouter_ballistic_gel} where the target code can be flashed and tested by the EM fault injection attacks. The setup is shown bellow.

 
 The lab setup described in this chapter includes a detailed walkthrough of the ChipWhisperer’s hardware components, including the capture board, target devices, and power supply configurations. Special attention is paid to key operational aspects such as signal synchronization, trigger alignment, and the calibration of timing parameters, all of which are vital to the success of glitch-based attacks. Safety considerations, especially when dealing with fault injection at the physical level, are also addressed to ensure both operator protection and equipment longevity.
 
 By documenting each step of the setup process, this chapter aims to provide a practical guide for researchers and practitioners who wish to replicate or build upon this work. Whether the goal is to explore new fault injection strategies, assess the resilience of embedded systems, or develop novel countermeasures, a clear understanding of the experimental foundation is critical. In doing so, this chapter contributes to the broader objective of advancing transparency, repeatability, and rigor in the field of hardware security research.
%\section*{Techniques and analysis: Side-Channel Attacks Using ChipWhisperer }
One of ChipWhisperer’s key strengths lies in its ability to perform highly controlled fault injection techniques, most notably voltage glitching and clock glitching. With voltage glitching, the platform can momentarily drop or distort the power supply to a target device, causing it to behave unpredictably, potentially skipping instructions or bypassing security checks. Clock glitching, on the other hand, involves inserting short, precisely-timed disturbances into the device’s clock signal. These glitches can disrupt the timing of operations, often triggering exploitable faults during critical processes like encryption or authentication. The fine-grained control over timing, width, and offset of these glitches makes ChipWhisperer exceptionally effective for exploring fault-induced vulnerabilities. Combined with its trace capture and analysis tools, these capabilities allow researchers to both induce and observe faults in real time, offering deep insight into how embedded systems respond under non-ideal operating conditions.

  \begin{minipage}[b]{0.3\linewidth}
    \centering
    \includegraphics[width=\linewidth]{images/cwlite_hardware.png}
    \smallskip
    CWLite with ARM 32-bit Target
  \end{minipage}
  \hfill
  \begin{minipage}[b]{0.3\linewidth}
    \centering
    \includegraphics[width=\linewidth]{images/cwhusky_hardware.png}
    \smallskip
   CWHusky with SAM4S Target
  \end{minipage}
  \hfill
  \begin{minipage}[b]{0.3\linewidth}
    \centering
    \includegraphics[width=\linewidth]{images/phy_hardware.png}
    \smallskip
    PhyWhisperer setup
  \end{minipage}
\section{ChipWhisperer Hardware Platform}
The ChipWhisperer platform is made up of specialized hardware designed to carry out side-channel analysis and fault injection on embedded devices. Its hardware setup is typically divided into two main parts: scope boards, which handle signal capture and glitching, and target boards, which run the code being tested. When used together, these components form a complete and flexible environment for testing the security of cryptographic systems and exploring hardware vulnerabilities.
\subsection{Scope Boards (Capture Hardware)}
Scope boards are at the heart of ChipWhisperer's side-channel capture capabilities. They serve as the interface between the host computer and the target device, enabling the collection of high-resolution power traces or electromagnetic emissions during cryptographic operations. These boards also provide fine-grained control over clock generation, triggering, and synchronization.

A popular example is the ChipWhispererLite(CWLite), an all-in-one board combining scope and target functionality, ideal for students and researchers. ChipWhispererHusky(CWHusky) offer higher sampling rates, more precise timing, and expanded features such as fault injection support.
Key features of scope boards:
\begin{itemize}
    \item Adjustable clock generation and distribution
    \item Trigger input/output for synchronization with DUT
    \item High-speed ADCs for capturing power consumption
    \item Communication interfaces (USB, serial, etc.)
\end{itemize}
Example usage scenario: Capturing power traces during AES encryption for differential power analysis (DPA).
\subsection{Target Boards (Device Under Test - DUT)}

Target boards, or Devices Under Test (DUTs), are embedded systems that execute the cryptographic or security-related code being analyzed. These boards are directly connected to the ChipWhisperer scope, allowing researchers to monitor power usage, apply glitches, and assess how the target responds to side-channel or fault injection attacks.

ChipWhisperer supports a variety of target boards suited for both beginners and advanced users. These targets range from basic microcontrollers to FPGA-based systems, making the platform highly flexible for different types of experiments.

Commonly used targets include:

\begin{itemize}
    \item CW308 UFO Board: A modular baseboard \cite{newae_cw308ufo} designed to support interchangeable target modules. It simplifies switching between different microcontrollers or chips without modifying the core setup.
    
    \item XMEGA Target: An Atmel XMEGA microcontroller board, commonly used for introductory experiments in side-channel analysis. It includes a reference AES implementation for easy testing.
    
    \item STM32F3 and STM32F4 Targets: ARM Cortex-M microcontrollers that are widely used in real-world applications. These targets are ideal for testing modern firmware under realistic fault or side-channel attack conditions.
    
    \item Artix-7 FPGA Target: A powerful platform \cite{newae_cw305} that allows the implementation and testing of custom cryptographic cores or hardware countermeasures. Suitable for advanced research in hardware security.
    
    \item SAM4S Target: Based on the ARM Cortex-M4 architecture, this board provides a more complex environment for evaluating side-channel resistance in higher-performance embedded systems.
\end{itemize}

Each of these targets includes features such as external clock input, adjustable power settings, and standard programming or debug interfaces (e.g., JTAG, SWD, UART). Together, they provide a robust environment for evaluating cryptographic implementations against a range of attack techniques.



\subsection{Integrated Target on ChipWhisperer-Nano}
The CWNano is a small and low-cost tool \ref{fig:nano_target} used to learn about hardware security. It has both a target microcontroller and a measurement unit on a single board, making it easy to use. The built-in STM32F030F4P6 microcontroller has 16~KB of flash and 4kB of SRAM, suitable for running simple cryptographic algorithms. The board supports power analysis using an 8-bit ADC with a sampling rate of up to 20~MS/s. It also features basic voltage glitching through a crowbar method. The Nano connects to a computer via USB and is controlled using the ChipWhisperer software suite. It does not support clock glitching and has a limited flash size, but it is ideal for beginners, students, and hobbyists exploring side-channel analysis and embedded security. More details about this can be found in the official documentation \cite{chipwhisperer_nano}.
\begin{figure}[h]
    \centering
    \includegraphics[width=0.8\linewidth, height=0.8\linewidth]{/home/cp07/mymtechthesis/thesisInParts/images/CWNANO.png}
    \caption{ChipWhisperer-Nano}
    \label{fig:nano_target}
\end{figure}



\subsection{ChipWhisperer-Lite}
The CWLite is a compact and low-cost hardware tool designed to help users learn about hardware security. It combines a power measurement system and a microcontroller target on a single board, making it easy to use. This device \cite{chipwhisperer_lite_capture} can capture power traces with high speed and allows for glitching attacks like clock or voltage glitches. It is mainly used to perform and study side-channel attacks, such as breaking encryption by analyzing power usage. The board connects to a computer through USB and works with open-source ChipWhisperer software. Because of its simplicity and low price, it is widely used in education and research.
\begin{figure}[h]
    \centering
    \includegraphics[width=0.7\linewidth, height=0.5\linewidth]{/home/cp07/mymtechthesis/thesisInParts/images/CWLITE.png}
    \caption{ChipWhisperer-Lite with 32-bit ARM target}
    \label{lite_target}
\end{figure}\newline
It is offered in four hardware options \cite{chipwhisperer_lite2} to suit different learning and testing needs. The first version is the CWLite-XMEGA, which includes both the capture hardware and an ATxmega128D4 microcontroller on a single board. The second option, CWLITE-ARM, features an STM32F3 ARM Cortex-M4 microcontroller instead of the XMEGA, also on a single board as shown in figure \ref{lite_target}. 
\begin{figure}[h]
    \centering
    \includegraphics[width=0.7\linewidth, height=0.6\linewidth]{/home/cp07/mymtechthesis/thesisInParts/images/cwlitw_2_part.png}
    \caption{ChipWhisperer-Lite 2-Part version connections}
    \label{fig:lite_2_part}
\end{figure}\newline
The CWLite 2-Part version as shown in figure \ref{fig:lite_2_part} separates the capture and target sections into two distinct boards, connected via SMA and 20-pin IDC cables. This modular design allows users to easily swap or upgrade target devices, enhancing flexibility for various testing scenarios. The capture board features a 10-bit ADC with a maximum sample rate of 105 MS/s, adjustable gain up to +55\,dB, and supports both voltage and clock glitching with fine-grained control. The target board includes an ATxmega128D4 microcontroller, suitable for implementing and analyzing cryptographic algorithms. This two-part configuration is ideal for users seeking a versatile and open-source platform for embedded hardware security research.\newline
The CWLite Standalone is a specialized capture board designed for side-channel power analysis and fault injection experiments. Unlike other variants, it does not include an integrated target microcontroller, providing flexibility to connect external targets via standard interfaces. It is ideal for researchers and educators who wish to interface with custom or third-party targets while utilizing ChipWhisperer's powerful capture and analysis capabilities.


\subsection{ChipWhisperer-Husky}
\begin{figure}[h]
    \centering
    \includegraphics[width=0.7\linewidth, height=0.7\linewidth]{/home/cp07/mymtechthesis/thesisInParts/images/cwhuskyannoted.png}
    \caption{ChipWhisperer-Husky with SAM4S target}
    \label{fig:husky_target}
\end{figure}
Together, the scope and target boards in the ChipWhisperer hardware ecosystem offer a comprehensive solution for conducting side-channel and fault injection attacks. Their modular, open-source nature makes them accessible to students, researchers, and engineers alike, and they are widely used in academic studies, industry evaluations, and security training environments.

\section{Software Environment}
The ChipWhisperer platform includes a flexible and scriptable software environment built primarily around Python and Jupyter Notebooks. This environment is well-suited for side-channel analysis and fault injection research, offering full control of the hardware and reproducibility. 

 Setting up the ChipWhisperer software environment involves installing the necessary tools to interface with the hardware, control experiments, and analyze captured data. This section outlines the steps required to install the ChipWhisperer software stack using the recommended method.
\subsubsection{For Windows :}
Installing ChipWhisperer on Windows is straightforward with the provided \.exe file \cite{chipwhisperer2025}. Once the download is complete, double-click the .exe file to launch the installer. If a security prompt appears, allow the installation to proceed. The ChipWhisperer Setup Wizard will open—follow the on-screen instructions by clicking "Next." Choose a destination folder or use the default option, then continue by clicking "Next" again. After the installation is finished, click "Finish" to complete the setup.
\begin{figure}[h]
    \centering
    \includegraphics[width=1\linewidth]{/home/cp07/mymtechthesis/thesisInParts/images/executable file.png}
    \caption{Chipwhisperer official GitHub page}
    \label{fig:windows_installation}
\end{figure}
The simplest way to start using ChipWhisperer and access its tutorials is to open the ChipWhisperer application. You can find it in the Start Menu, the installation directory, or on your desktop if you chose to create a shortcut during setup.

\subsubsection{for Linux:}

To install ChipWhisperer on a Linux-based system, follow these steps \cite{chipwhisperer_linux_install}:\\

Step 1: Update System Packages\\
\begin{lstlisting}[language=bash]
sudo apt update && sudo apt upgrade
\end{lstlisting}

Step 2: Install Required Dependencies\\
\begin{lstlisting}[language=bash]
sudo apt install libusb-dev gcc-arm-none-eabi make git
 avr-libc gcc-avr avr-libc libusb-1.0-0-dev usbutils
python3 python3-venv
\end{lstlisting}

Step 3: Clone the Repository\\
\begin{lstlisting}[language=bash]
cd ~/
git clone https://github.com/newaetech/chipwhisperer
cd chipwhisperer
\end{lstlisting}

Step 4: Set Up Python Virtual Environment\\
\begin{lstlisting}[language=bash]
python3 -m venv ~/.cwvenv
source ~/.cwvenv/bin/activate
\end{lstlisting}

Step 5: Install USB Rules\\
\begin{lstlisting}[language=bash]
sudo cp 50-newae.rules /etc/udev/rules.d/
sudo udevadm control --reload-rules
\end{lstlisting}

Step 6: Add User to Required Groups\\
\begin{lstlisting}[language=bash]
sudo groupadd -f chipwhisperer
sudo usermod -aG chipwhisperer $USER
sudo usermod -aG plugdev $USER
\end{lstlisting}

Step 7: Initialize Submodules and Install Python Packages\\
\begin{lstlisting}[language=bash]
git submodule update --init jupyter
python -m pip install -e .
python -m pip install -r jupyter/requirements.txt
\end{lstlisting}

Step 8: Reboot System\\
Once the setup is complete, reboot your system to apply changes.

Step 9: Launch Jupyter Notebooks\\
\begin{lstlisting}[language=bash]
cd ~/chipwhisperer
jupyter notebook .
\end{lstlisting}

\subsection{Directory Structure}
The ChipWhisperer framework is a comprehensive toolkit for side-channel power analysis and glitching attacks. Its directory structure is designed to support both hardware interfacing and educational experimentation. Understanding this structure is key to navigating the project efficiently, especially when working with Jupyter notebooks or compiling firmware.

Here’s a breakdown of the most relevant parts:

\subsection*{jupyter/ Directory :}
One of the most user-friendly components of the ChipWhisperer project is the \texttt{jupyter/} directory. This folder contains a wide range of Jupyter Notebooks designed to guide users through hands-on tutorials and experiments.

These notebooks provide interactive lessons in topics such as: Side-channel analysis, Cryptographic attacks, Glitching techniques
The content within the \texttt{jupyter/} directory is well-organized into subfolders, such as:

The Jupyter directory in ChipWhisperer includes several folders designed to support learning and experimentation. The \texttt{demos} folder contains simple, hands-on notebooks meant for beginners. These notebooks introduce key concepts like power analysis and fault injection through guided examples. The \texttt{courses} folder offers more structured content, similar to academic classes, with lecture-style explanations and lab exercises for deeper learning. The \texttt{setupscripts} folder provides example scripts and configurations that demonstrate different experiments and usage scenarios for ChipWhisperer hardware. Together, these resources help users understand and explore various aspects of side-channel analysis and hardware security.
\subsection*{firmware/ Directory:}
The \texttt{firmware/} directory in the ChipWhisperer repository contains all the source code necessary to build firmware for a variety of supported target devices. This firmware is essential for running side-channel analysis and fault injection experiments, as it enables precise control and repeatable behavior during cryptographic operations on the Device Under Test (DUT).

The \texttt{firmware/} folder is organized into several subdirectories that correspond to specific microcontroller families, target boards, or cryptographic examples. \texttt{firmware/mcu} contains Cryptographic firmware examples such as AES and RSA with SimpleSerial protocol support for host-target communication.

\subsection{Scope API}
The \texttt{scope} object in ChipWhisperer is used to manage the capture and glitching operations of the hardware.The official documentation \cite{chipwhisperer_scope_api} provides a comprehensive overview of the scope API, detailing how to configure the hardware, capture traces, and perform glitching operations. The scope API is designed to be intuitive and flexible, allowing users to easily set up their experiments and interact with the ChipWhisperer hardware.

To create a \texttt{scope} object,  the 
\texttt{chipwhisperer.scope()} function is the easiest approach to use. This function connects to a ChipWhisperer device and returns an instance of the appropriate scope object:

\begin{lstlisting}[language=Python]
import chipwhisperer as cw
scope = cw.scope()
\end{lstlisting}

There are two types of \texttt{scope} : OpenADC for \texttt{CWLite/Husky} and CWNano for \texttt{CWNano}. The choice of scope depends on the specific hardware being used.
The \texttt{scope} object provides various properties and methods to configure the hardware, capture traces, and perform glitching operations. Here are some of the key properties and methods:
\subsubsection{\textbf{\texttt{scope.adc.samples} :}}
This property sets the number of samples to capture by the  ADC (Analog-to-Digital Converter). It is useful for determining the length of the captured trace.
maximum number of samples for Lite is 244000 and for Husky is 131070

\subsubsection{\textbf{\texttt{scope.adc.timeout }:}}
  
Specifies the maximum duration (in seconds) the ADC will wait for a trigger signal before aborting the capture. This prevents indefinite waiting during a capture session.
\subsubsection{$scope.clock.adc\_src :$}

Determines the clock source for the  ADC  module. Generally it will be $clkgen\_x1$ or $clkgen\_x4$ which correspond to different clock frequencies.
\subsubsection{\textbf{\texttt{glitch.clk\_src }:}}

This setting selects the clock source used by the glitch module's DCM (Digital Clock Manager). The DCM determines the timing of when glitches are produced.
The available clock sources are:

\begin{itemize}
  \item clkgen: Uses the output from the internal clock generator. \textit{Note: This option is not supported on Husky.}
  \item pll: Uses the on-board Phase-Locked Loop (PLL) available on Husky devices. \textit{Note: This is specific to Husky.}
\end{itemize}

Selecting the correct clock source is important for ensuring that glitches are timed correctly with the target device’s operation.
\subsubsection{$scope.glitch.output :$}
This setting controls the type of signal produced by the glitch module. It defines how the regular clock and glitch pulses are combined to create the final output signal.
The available output modes are:

\begin{itemize}
  \item clock\_only: Outputs only the normal clock signal, with no glitches applied.
  \item glitch\_only: Outputs only the glitch pulses, ignoring the clock signal entirely.
  \item clock\_or: The output is high when either the clock or the glitch pulse is high.
  \item clock\_xor: The output is high only when the clock and glitch pulse differ.
  \item enable\_only: The output stays high for a set number of full clock cycles. In this mode, \texttt{width} and \texttt{width\_fine} have no effect, but \texttt{offset} and \texttt{offset\_fine} are still used.
\end{itemize}

These modes are chosen based on the type of glitching attack:
\begin{itemize}
    \item For clock glitching, use \texttt{clock\_or} or \texttt{clock\_xor}.
    \item For oltage glitching, use \texttt{glitch\_only} or \texttt{enable\_only}.
  \end{itemize}

\subsubsection{$scope.glitch.trigger\_src :$}    
The glitch module supports four trigger modes:

\begin{itemize}
    \item Continuous: Glitches are triggered constantly. Parameters like \texttt{ext\_offset}, \texttt{repeat}, and \texttt{num\_glitches} have no effect.
    
    \item Manual: Glitches are triggered manually via \texttt{manual\_trigger()} or \texttt{scope.arm()}. Only \texttt{repeat} is relevant; others are ignored.
    
    \item ext\_single: Triggers once per arming when a condition is met. Ignores further triggers until re-armed. \texttt{ext\_single} provides a controlled, one-time glitch per arming cycle based on an external condition—making it useful for repeatable and precise fault injection experiments.
    
    \item ext\_continuous: Triggers glitches repeatedly whenever the trigger condition is met, regardless of arm status.
\end{itemize}
\subsubsection{$scope.glitch.repeat :$}
\texttt{repeat} is a setting that controls how many glitch pulses are generated per trigger.

\begin{itemize}
    \item If the output mode is \texttt{glitch\_only}, \texttt{clock\_or}, or \texttt{clock\_xor}, each count in \texttt{repeat} represents a separate glitch pulse.
    
    \item If the output mode is \texttt{enable\_only}, the glitch is a single pulse that lasts for \texttt{repeat} clock cycles.
    
    \item This helps in creating stronger glitches, especially for voltage glitching.
    
    \item On \textbf{CW-Husky}, if multiple glitches are used (\texttt{num\_glitches > 1}), \texttt{repeat} should be a list with one value per glitch. Each value must be $\leq \texttt{ext\_offset}[i+1] + 1$.
    
    \item On \textbf{CW-Lite/Nano}, only one glitch is supported, so \texttt{repeat} is a single integer.
    
    \item The value of \texttt{repeat} must be in the range [1, 8192], and it has no effect in \texttt{continuous} mode.
\end{itemize}
\subsubsection{\textbf{\texttt{scope.glitch.width} :}} 
The \texttt{width} property defines how wide a single glitch pulse is and can be set using either a float or an integer. Its meaning depends on the type of hardware used. For CWHusky, the width is measured in phase shift steps. A value of 0 gives the smallest pulse, while the maximum is at half the total number of phase shift steps. Negative values are allowed and are interpreted as wrapping around the phase shift range, so $-x$ is treated the same as $\texttt{total\_steps} - x$. The values also wrap around when going beyond the maximum.

For other devices like CWLite or CWNano, the width is given as a percentage of one clock cycle. You can set the pulse from about $-49.8\%$ to $+49.8\%$ of the cycle. A width of $0\%$ will not work reliably. Negative widths behave like their positive counterparts but are applied to the opposite half of the clock cycle. This setting has no effect if the output mode is set to \texttt{enable\_only}.

\subsubsection{\textbf{\texttt{scope.glitch.offset }:}}
The \texttt{offset} property sets the delay between the rising edge of the clock and the start of the glitch pulse. It can be a float or an integer, and its meaning depends on the hardware used.

For CWHusky, the offset is measured in phase shift steps. An offset of 0 means the glitch pulse starts exactly at the rising edge of the clock. At half of \texttt{scope.glitch.phase\_shift\_steps}, the glitch starts at the falling edge of the clock. We can use negative values, and $-x$ is treated the same as \texttt{scope.glitch.phase\_shift\_steps} $- x$. The value also wraps around, so $+x$ is the same as \texttt{scope.glitch.phase\_shift\_steps} $+ x$. 

For other devices like CWLite or CWPro, the offset is given as a percentage of one clock period. The glitch can start anywhere from $-49.8\%$ to $+49.8\%$ of the clock cycle. This allows us to move the glitch to any point within the cycle.

\subsubsection{$scope.glitch.ext\_offset :$}
The \texttt{ext\_offset} property defines how many clock cycles the glitch module should wait after receiving a trigger before generating a glitch pulse. This delay is useful when you want to insert the glitch at a specific point in the target device’s operation, such as during the execution of a particular instruction. On CW-Lite and CW-Pro, multiple glitches are not supported, so \texttt{ext\_offset} is simply an integer representing the delay after the trigger to the single glitch. This setting has no effect if the trigger source is set to \texttt{manual} or \texttt{continuous}.

The \texttt{offset} property, on the other hand, controls the position of the glitch within a single clock cycle. While \texttt{ext\_offset} determines \emph{when} the glitch should start in terms of clock cycles after the trigger, \texttt{offset} adjusts the glitch \emph{within} a chosen clock cycle—allowing very precise control relative to the rising or falling edge of the clock signal. In CW-Husky, \texttt{offset} is measured in phase shift steps, and in CW-Lite or CW-Pro, it is expressed as a percentage of one clock period. Together, these two properties provide both coarse and fine control over glitch timing.


\subsubsection{$scope.arm() :$}
The \texttt{arm()} function prepares the scope to start capturing data or performing a glitch when a trigger is received. This step is required before any capture or glitch attempt can begin. If the scope is set to \texttt{ext\_single} mode, it will remain idle until it is armed and a valid trigger event occurs. Without calling \texttt{arm()}, the scope will not respond to trigger signals.

\subsubsection{$scope.capture() :$}
The \texttt{capture(poll\_done=False)} function starts the process of capturing a trace. Before using this function, the scope must be armed using \texttt{arm()}. Once called, it waits for a trigger event. When the trigger occurs or a timeout is reached, the function stops the capture, disarms the scope, and retrieves the recorded data.


\subsubsection{$scope.get\_last\_trace() :$}
The \texttt{get\_last\_trace(as\_int=False)} function returns the most recent trace captured by the scope. By default, it provides the data as a NumPy array of floating-point values scaled between \texttt{-0.5} and \texttt{0.5}. If the parameter \texttt{as\_int} is set to \texttt{True}, the function returns the trace as raw integer values, which are the direct outputs from the ADC of the ChipWhisperer device. The resolution of these values depends on the hardware: for example, the ChipWhisperer-Lite uses a 10-bit ADC, the Nano uses 8-bit, and the Husky can use either 8-bit or 12-bit ADC data.

\subsection{Target API}
The \texttt{target} object in ChipWhisperer is used to manage the device under test (DUT) and perform operations such as loading firmware, executing commands, and capturing traces. It provides a high-level interface for interacting with the target device.
ChipWhisperer provides two classes for UART communication:

\begin{itemize}
    \item Simple Serial Target (default)
    \item Simple Serial V2 Target
\end{itemize}

The most straightforward way to create a target object in ChipWhisperer is by using the \texttt{cw.target} function. Here is a simple example:

\begin{lstlisting}[language=Python]
import chipwhisperer as cw
scope = cw.scope()
try:
    if SS_VER == "SS_VER_2_1":
        target_type = cw.targets.SimpleSerial2
    else:
        target_type = cw.targets.SimpleSerial
except:
    SS_VER="SS_VER_1_1"
    target_type = cw.targets.SimpleSerial

try:
    target = cw.target(scope, target_type)
\end{lstlisting}

This code initializes the ChipWhisperer scope and sets up the Simple Serial target.

some useful methods and properties of the \texttt{Simple Serial V2 Target} object include:
\subsubsection{$target.flush() :$}
The \texttt{flush()} function clears all data currently stored in the serial buffer. This is useful when you want to discard any previous or unwanted serial communication data before starting a new operation. It helps ensure that the buffer only contains fresh data relevant to the current task.

\subsubsection{$target.simpleserial\_write(cmd, data) :$}
This function sends a command and associated data to a target device using the SimpleSerial protocol. This function is typically used when communicating with firmware that implements cryptographic functions such as AES encryption. 

The \texttt{cmd} parameter is a one-character string that specifies the type of command. For example, using \texttt{'p'} tells the device to encrypt the given plaintext, while \texttt{'k'} sets the encryption key. These special cases internally map to specific command and sub-command values expected by the firmware.

The \texttt{data} parameter is a \texttt{bytearray} containing the actual data to be sent with the command. If no data is provided, a single byte \texttt{[0x00]} is sent by default. The optional \texttt{end} argument is reserved but not used in this implementation.

\vspace{0.3cm}
\noindent
\textbf{Example:} To send a 16-byte plaintext to be encrypted using the AES block cipher, one might use the following:

\begin{verbatim}
target.simpleserial_write('p', bytearray([0x00]*16))
\end{verbatim}

This line sends a block of 16 zero bytes to the device with the command \texttt{'p'}, which typically triggers AES encryption of the plaintext using a previously loaded key.

\subsubsection{$target.simpleserial\_read\_witherrors(cmd, length) :$}
The \texttt{simpleserial\_read\_witherrors()} function is used to read data from a device over a serial connection, especially when doing fault injection or glitch experiments. It tries to receive a full response from the device that includes a command, some data, and a special ending byte. If the response is incomplete, contains errors, or takes too long, the function will try one more time with a longer timeout to get whatever data is available. This is helpful when the target device behaves in unexpected ways due to glitches. The result is returned as a dictionary, which includes whether the response was valid, the decoded data if successful, the full raw output, and any return values if needed.

\vspace{0.3cm}
\noindent
\textbf{Example:} Suppose we expect a 16-byte result from a previous command, such as an AES ciphertext. We can read it like this:

\begin{verbatim}
response = simpleserial_read_witherrors(cmd='r', pay_len=16)
if response['valid']:
    print("Received:", response['payload'])
else:
    print("Invalid response. Raw output:", response['full_response'])
\end{verbatim}

In this example, the function tries to read 16 bytes from a response starting with the command \texttt{'r'}. If it fails due to glitches or timing issues, it still gives the raw data so we can analyze what went wrong.

\subsubsection{$target.simpleserial\_wait\_ack(cmd, timeout=1) :$}
This function is used to wait for an acknowledgment (ack) or error message from the target device after sending a command. This is useful to confirm whether the target received and understood the previous instruction. You can set a timeout in milliseconds, which defines how long to wait for the ack. If the timeout is set to 0, the function will wait indefinitely until it receives a response. If no ack is received within the given time, the function returns \texttt{None}. Otherwise, it returns a code that indicates the result of the command.

